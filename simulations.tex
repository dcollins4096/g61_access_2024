\section{Simulations}
\label{sec.simulations}

Our proposed simulations are in two suites.  The first suite is \red{12}
simulations of driven turbulence.  These simulations, while highly idealized,
will examine the detailed relation between the state of the ISM and its
contribution to polarization foregrounds.  The second suite is 9 full galaxy
simulations.  These lack resolution, but will serve as preliminary runs to
ensure the initial conditions and subgrid models will behave at scale before
performing the larger scale simulations.  \red{We additionally propose time for
analysis of both allocations}

\subsection{Turbulence Simulations}

The ISM is dominated by turbulence, and driven turbulent boxes are the most
efficient way to separate driving scale and dissipation scale in spectral space
to simulate turbulence in a self consistent way.  
The driving scale is a boundary condition provided
by the scientist, and the dissipation scale is a combination of
resolution and method.  Increasing resolution increases the separation between
these boundary conditions and allow us to explore the true nonlinear physics of
fluid dynamics.  The preliminary simulations of Stalpes et al showed
power law behavior of $E$ and $B$ with appropriate slopes, but only in a limited range of spectral
space, in some cases negligible.  Further, we were unable to proceed to Mach
numbers higher than 6 due to the limited spatial resolution.  
High Mach number simulations result in large density contrast, which is
equivalent to high spatial contrast, which is only realizable with high
resolution.  In Stalpes et al, we predict that the typical Mach and \alf\ Mach
numbers are 4.7 and 1.5.  Our high resolution simulations target this parameter
space.  But thee true ISM is multiphase, with many patches exhibiting a wide
range of temperatures, densities, sound speeds, and Mach numbers.  Thus we also
propose a suite of 12 simulations, where we verify the results of Stalpes et al,
as well as extending to higher Mach numbers, and anticipating the success of our
\emph{Frontera} simulations.  

We will simulate for 5 dynamical times \red{FLESH THIS OUT}

\subsection{Galaxy Simulations}

Our ultimate desire is to simulate the mid-plane of a galaxy with 1pc resolution
while also simulating the CGM to roughy 1Mpc. This will give us a realistic
picture of the small scale CMB foregrounds where the interesting signal lies, as well as the large scale
properties that are necessary for its removal.  This will proceed in several
steps.  


\red{MORE FLESH SHIT I DIDNT FINISH}
