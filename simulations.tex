\section{Simulations}
\label{sec.simulations}

Our proposed simulations are in two suites.  The first suite is \red{12}
simulations of driven turbulence.  These simulations, while highly idealized,
will examine the detailed relation between the state of the ISM and its
contribution to polarization foregrounds.  The second suite is 9 full galaxy
simulations.  These lack resolution, but will serve as preliminary runs to
ensure the initial conditions and subgrid models will behave at scale before
performing the larger scale simulations.  We additionally propose time for
analysis of all of the simulation products

\subsection{Turbulence Simulations}

The ISM is dominated by turbulence, and driven turbulent boxes are the most
efficient way to separate driving scale and dissipation scale in spectral space
to simulate turbulence in a self consistent way.  
The driving scale is a boundary condition provided
by the scientist, and the dissipation scale is a combination of
resolution and method.  Increasing resolution increases the separation between
these boundary conditions and allow us to explore the true nonlinear physics of
fluid dynamics.  The preliminary simulations of Stalpes et al showed
power law behavior of $E$ and $B$ with appropriate slopes, but only in a limited range of spectral
space, in some cases negligible.  Further, we were unable to proceed to Mach
numbers higher than 6 due to the limited spatial resolution.  
High Mach number simulations result in large density contrast, which is
equivalent to high spatial contrast, which is only realizable with high
resolution.  In Stalpes et al, we predict that the typical Mach and \alf\ Mach
numbers are 4.7 and 1.5.  Our high resolution simulations target this parameter
space.  But thee true ISM is multiphase, with many patches exhibiting a wide
range of temperatures, densities, sound speeds, and Mach numbers.  Thus we also
propose a suite of 12 simulations, where we verify the results of Stalpes et al,
as well as extending to higher Mach numbers, and anticipating the success of our
\emph{Frontera} simulations.  

We will simulate for 5 dynamical times, where we define $t_{dyn}=0.5 L/\mach$.
This is the time for the driving pattern (which has a size of $0.5 L$, where L
is the box size) to turn over once.  Frames separated by this time are
essentially uncorrelated.  Our simulations take about 2 $t_{dyn}$ to develop a
statistically steady state that we can average over.  We then average over 3
$t_{dyn}$ to develop statistics.  We have determined that this is the minimum
amount of time to average to obtain a clean signal;  more would be better, but
that comes with increased cost.

The resolution of these simulations will be $1024^3$.  
This will bring a factor of 2 increase in the inertial range of our spectra over
our preliminary results, strengthening the conclusions we can draw
significantly.  A further increase to $2048^3$ would further improve the
predictive power, but this is far too expensive to do measure the behavior of
all of the parameter space of interest.  

\subsection{Galaxy Simulations}

Our ultimate desire is to resolve the mid-plane of a galaxy with 1pc resolution
while also simulating the CGM to roughy 1Mpc. This will give us a realistic
picture of the small scale CMB foregrounds where the interesting signal lies, as well as the large scale
properties that are necessary for its removal.  This will proceed in several
steps.  For each step, the basic layout of the grids is a thin region of high
resolution, covering $\pm100$pc where the bulk of the dust lies, fully covering
the disk in a $20kpc\times 20kpc$ region.  This is a very high aspect ratio
layer of refinement.  The next coarser level decreases the aspect ratio, while
covering the finest grid.  At each level, we strive to keep roughly the same
amount of work, and arrange the refinement pattern in a series of regular
blocks.  This will eliminate much of the overhead and inefficiency of a dynamic
AMR simulation.  

The first step is to perform the simulations proposed here, which will have an
outer box of 819,200 pc and a fine resoltion of 12.5 pc.  These will be stacks
of 8 levels of roughly $256^3$ zones per level.  These simulations will
ensure the subgrid models (chemistry and star formation) are behaving properly
and to deduce a first look at galactic magnetic fields.  We will run 3 fiducial runs,
wherein we vary the behavior of the magnetic field to model magnetic generation
mechanisms.  The first run has no magnetic field, as a control; the second will
have a weak uniform field, to provide a seed for the dynamo; and the third
fiducial run will allow supernovae to inject magnetic fields into the
simulation.  These three simulations will be repeated three times, for a total of
9 galaxies, wherein we vary the CGM model and star formation recipies.  

The second step of our campaign to simulate galaxies will be performed on
\emph{Frontera}, where we double the resolution. These simulations will be a
stack of 8 levels of roughly $512^3$ at each level, and resolve the midplane to
6.25pc. Due to the expense, there are only three such simulations, where we will
vary the magnetic production (none, uniform, and supernovae) as in the first
generation.

The third and fourth steps are more ambitious, doubling the resolution two more
times to $1024^3$ per level and finally $2048^3$ per level.  Once we are
successful with the currently allocated galaxy towers and the $2048^3$ driven
turbulence, we will be ready to combine the two.  


