
We are requesting \request\ on Frontera and \disk\ on the Ranch archival system in order to perform several studies of
the interstellar medium (ISM).  This is part of an effort to observe
gravitational waves from the big bang, imprinted in the cosmic microwave background (CMB).  These observations are hampered by the
dust and plasma in our own galaxy, which outshines the CMB.  We will simulate
the plasma in the ISM in two ways to characterize its observational properties.
One study will be high resolution simulations of plasma turbulence, and the
other suite will be simulations of isolated galaxies.  The turbulence
simulations give us a detailed study of a typical patch of the ISM, while the
galaxy simulations will study the large scale structure.  The disk requests
represents the newly created simulations as well as our archive of turbulence
and star formation simulations.

\section{Background}

Gravitational waves generated at
the beginning of the Universe leave an imprint in the polarization of the cosmic
microwave background (CMB).  The CMB is the oldest light observable in the
Universe, and gives us a snapshot of its very early stages.
Observing the polarization of the CMB will give us a measurement of
gravitational waves that were produced at the formation of the Universe.
Unfortunately, the dusty plasma in our own Milky Way also produces polarized
light that is brighter than the primordial signal.  Our simulations will study
the polarization properties of the light from the plasma in Milky Way type
galaxies.  Understanding this polarization is essential for its removal from
future measurements of the CMB.

The Planck satellite \citep{Planck18xi}  measured the polarization of the entire sky.  Polarization,
having direction, needs two quantities for its description. The most
useful option are the parity-even $E$-mode and parity-odd $B$-mode.  
The only source of $B-$ mode polarization in the CMB are gravitational waves,
so searching for $B-$modes in the CMB will prove quite profitable.  However, the
dusty, magnetized plasma in our own galaxy also produces
$B-$mode polarization.  So this must be understood so it can be removed.  

The Planck measurements 
found that both are $E-$ and $B-$mode are relatively uniform power-laws in wavenumber, 
with spectra given by
\begin{align}
C_\ell^{EE} = A^{EE} k^{\alpha^{EE}},
\end{align}
for $E-$mode and a similar expression for $B-$mode.  They found that the slopes
are roughly equal, $\alpha^{EE}\sim\alpha^{BB}\sim-2.5$, and the $B-$mode have
roughly half the power of the $E-$mode, so $A^{EE}/A^{BB}\sim 2$.  This signal
has three components: the CMB signal that we are interested in; thermal dust
emission near the Sun; and high altitude synchrotron electrons.  Both the dust
and the synchrotron electrons are aligned by the magnetic field that threads the
galaxy, and the light they emit is thus polarized.  

A curious finding in the Planck data is a correlation between the total
signal, $T-$mode, which is parity-even, and the parity-odd $B-$mode.  In principle, these should be
uncorrelated, but the Planck satellite measured a non-zero correlation at the
5\% level.  This $TB$ may come from some unseen property of the plasma, or it
could come from large scale structure in the local magnetic field.  

With a suite of MHD turbulence simulations run on \emph{Stamped 2}, we have shown (Stalpes et al 2023, in prep) that the slopes of the polarization
are directly related to the state of the turbulent gas, specifically the r.m.s.
velocity and magnetic field strength.  Our suite of 21 simulations 
varied the Mach number, $\mach=v/c_s$, the ratio of
r.m.s velocity to sound speed, and \alf\ mach number, $\alfmach=v/v_A$, where
$v_A$ is the speed of magnetic disturbances, in a suite of driven turbulence
simulations.  Increasing \mach\ makes for more
filamentary structure in the cloud, and slopes consistent with what Planck
observed, and increasing magnetic field suppresses small scale structure.  Our suite employed \mach = (\half,1,2,3,4,5,6) and \alfmach=(\half,1,2).  
We perform synthetic observations of polarized emission of the gas, and find
that both $E$ and $B$ are distributed like a power law, and their slopes are
roughly
linearly related to \mach\ and \alfmach.
After inverting our linear relationship,  we predict that the observed values of $\alpha^{EE}$ and
$\alpha^{BB}$ are best matched by plasma with \mach=4.7 and \alfmach=1.5. While
the ISM has many patches of gas with a variety of parameters, our preliminary
this represents a ``typical'' patch of sky.  Our new simulations will verify
this prediction, and provide high resolution mock datasets with similar
spectral properties to the observed microwave sky.
The
study presented in Stalpes et al were runs were at a modest resolution of
$512^3$ with limited inertial range.   Our simulations will improve the
resolution to $2048^3$ which will improve the fidelity of the results, and test
our prediction of the typical \mach\ and \alfmach\ of the local galaxy.

The results from Stalpes et al (2023) show that turbulence alone can
account for a $TB$  correlation only at the 2\% level at best.   These were
relatively low resolution studies, it is possible that the turbulent correlation
is lower.  This would imply  the $TB$
correlation seen in the sky comes from large scale features in the dust and magnetic field
morphology.  To properly test this hypothesis, we will 
perform simulations of Milky Way sized disk galaxies 
that have high resolution where the dust is, and also capture the circumgalactic
medium (CGM) around the galaxy.  Including both the disk and the CGM is
essential as the two form a coupled system, and at present it is not clear how
far above the galaxy the synchrotron electrons reside.
The dust and star formation
primarily resides in a thin disk, with a scale height of 100 pc.  
The interstellar medium is a dynamical environment,
driven by supernova explosions and star formation.  
These explosions expel
material from the disk into the CGM, which then cools and falls back to the disk,
providing fuel for more star formation.  Thus to properly simulate the
morphology of the magnetic field in and around a galaxy, one must also properly model
the surrounding CGM.
 All of these pieces are important for a complete picture of the
polarized sky.

We propose two suites of simulations.  The first is a continuation of the
moderate resolution turbulent boxes presented in Stalpes et al (2023).  In that
study, we predict that a Mach number of 4.7 and an \alf\ Mach number of 1.5
reproduce the sky.  We will perform two simulations at $2048^3$.  These
simulations will test our prediction, and provide high quality datasets that
reproduce the salient features of the sky that can be used to test foreground
removal algorithms algorithms and ISM models.  While these simulations are quite
large, the excellent scaling of our code on Frontera and our years of experience guarantee success.
The second suite of simulations
is a set of three isolated galaxies that will simultaneously provide high
resolution on the mid plane, and properly resolve the environment and boundary
conditions of the galaxy.  These will be used initially to test CMB foreground
models, and will also have many applications beyond the CMB.  These galaxies
will be a stack of nine fixed resolution levels at roughly $512^3$ per level,
with the finest level concentrated in a thin disk.  Employing fixed
resolution and consistent grid size allows us to optimize the performance of
the simulations.

We will introduce the code in Section \ref{sec.methods}, outline and motivate
the simulation design in Section \ref{sec.simulations}, and proved a detailed
accounting of the performance of the simulations in Section \ref{sec.request}.
The performance and scaling of the code can be found in Section
\ref{sec.scaling}, and a short account of our previous success in simulating
turbulence and star formation can be found in Section \ref{sec.previous}.
