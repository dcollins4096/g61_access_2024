\section{Request Details}
\begin{table} \begin{center} \caption{Detailed accounting for each suite of simulations. The galaxy
simulations are broken down by level for one galaxy.  $N_Z=1.1\sci{9}$ for the turbulence
simulations.  We will use 512 cores for
the galaxy simulations and 4096 cores for the turbulence simulations, using the
whole node (128 cores) for all simulations.}
 \label{table3}                                                                                                               
\begin{tabular}{       l               r               r               r               r               r               r      }
  \hline                                                                                                               
           suite       &  $\ell$       &   $N_Z$       &$\Delta T$[s]       & $dT$[s]       & $\zeta$       &      SU     \\
  \hline                                                                                                               
          Galaxy      &        8       &4.1\sci{7}       &3.1\sci{10}       &6.3\sci{16}       &5.0\sci{4}       &4.7\sci{5}     \\
          Galaxy      &        7       &3.6\sci{7}       &6.2\sci{10}       &6.3\sci{16}       &5.0\sci{4}       &2.0\sci{5}     \\
          Galaxy      &        6       &4.1\sci{7}       &1.2\sci{11}       &6.3\sci{16}       &5.0\sci{4}       &1.2\sci{5}     \\
          Galaxy      &        5       &1.7\sci{7}       &2.5\sci{11}       &6.3\sci{16}       &5.0\sci{4}       &2.4\sci{4}     \\
          Galaxy      &        4       &1.7\sci{7}       &4.9\sci{11}       &6.3\sci{16}       &5.0\sci{4}       &1.2\sci{4}     \\
          Galaxy      &        3       &1.7\sci{7}       &9.9\sci{11}       &6.3\sci{16}       &5.0\sci{4}       &6.0\sci{3}     \\
          Galaxy      &        2       &1.7\sci{7}       &2.0\sci{12}       &6.3\sci{16}       &5.0\sci{4}       &3.0\sci{3}     \\
          Galaxy      &        1       &1.7\sci{7}       &3.9\sci{12}       &6.3\sci{16}       &5.0\sci{4}       &1.5\sci{3}     \\
          Galaxy      &        0       &1.7\sci{7}       &7.9\sci{12}       &6.3\sci{16}       &5.0\sci{4}       &7.4\sci{2}     \\
  \hline                                                                                                               
                       &               &               &               &                Total SU & 1 galaxy       &8.3\sci{5}     \\
                       &               &               &               &                Total Disk & 1 galaxy       &8.8\sci{2}     \\
                       &               &               &               &        \bf{Total SU} & \bf{       6}\bf{~galaxies}&5.0\sci{6}     \\
                       &               &               &               &        \bf{Total Disk} &\bf{       6}\bf{~galaxies}&5.3\sci{3}     \\
  \hline                                                                                                               
           suite       & $\mach$       &$\alfmach$       &$\Delta T$       &    $dT$       & $\zeta$       &      SU     \\
  \hline                                                                                                               
        Turbulence      &        2       &    0.75       &     1.3       &3.3\sci{-6}       &2.4\sci{5}       &4.7\sci{5}     \\
        Turbulence      &        2       &     1.5       &     1.3       &4.4\sci{-6}       &2.4\sci{5}       &3.6\sci{5}     \\
        Turbulence      &        2       &       3       &     1.3       &5.2\sci{-6}       &2.4\sci{5}       &3.0\sci{5}     \\
        Turbulence      &      4.7       &    0.75       &     0.5       &1.6\sci{-6}       &2.4\sci{5}       &4.2\sci{5}     \\
        Turbulence      &      4.7       &     1.5       &     0.5       &2.1\sci{-6}       &2.4\sci{5}       &3.1\sci{5}     \\
        Turbulence      &      4.7       &       3       &     0.5       &2.6\sci{-6}       &2.4\sci{5}       &2.5\sci{5}     \\
        Turbulence      &        8       &    0.75       &     0.3       &9.6\sci{-7}       &2.4\sci{5}       &4.0\sci{5}     \\
        Turbulence      &        8       &     1.5       &     0.3       &1.3\sci{-6}       &2.4\sci{5}       &2.9\sci{5}     \\
        Turbulence      &        8       &       3       &     0.3       &1.6\sci{-6}       &2.4\sci{5}       &2.4\sci{5}     \\
        Turbulence      &       12       &    0.75       &     0.3       &6.5\sci{-7}       &2.4\sci{5}       &4.8\sci{5}     \\
        Turbulence      &       12       &     1.5       &     0.2       &9.0\sci{-7}       &2.4\sci{5}       &2.9\sci{5}     \\
        Turbulence      &       12       &       3       &     0.2       &1.1\sci{-6}       &2.4\sci{5}       &2.3\sci{5}     \\
  \hline                                                                                                               
                       &               &               &               &               &\bf{Total SU}       &3.0\sci{6}     \\
                       &               &               &               &               &\bf{Total Disk}       &1.6\sci{3}       
\end{tabular} \end{center} \end{table}                                                                                                                


\label{sec.request}
The cost for each simulation is found as
\begin{align}
SU &= t_{wall} N_C\\
t_{wall} &= \frac{\sum_{\ell} N_Z N_U}{\zeta} \frac{1}{N_C},
\end{align}
where $SU$ is the total cost in core-hours; $t_{wall}$ is the run time; 
 $N_Z$ is the number of zones per level $\ell$; $N_U$ is the number of
updates per level; $\zeta$ is
the performance measured in (zone-updates)/(processor-second); and $N_C$ is the
total number of cores.  The cost, $zeta$, is determined in the scaling document.
Total disk usage is found as
\begin{align}
	Disk = 8 N_Z N_F N_D \rm{bytes},
\end{align}
where each of $N_Z$ zones contains $N_F$ fields and we store $N_D$ dumps for
each run.  The values of $N_N$, $N_C$, $N_F$, $N_D$ and $\zeta$ can be found in
$N_F$ = (20, 27) and $N_D$=(10,20) for the (turbulence, galaxy) simulations,
respectively.   The values of $N_Z$ and $N_U$ are outlined in Table
\ref{table3} and the rest of this section.

The number of zones, $N_Z$, is computed from the problem geometry.  All of these
simulations will use fixed or static resolution, so $N_Z$ is known.  
For the \emph{turbulence} simulations, $N_Z=1024^3$.  For the galaxy
simulations, $N_Z$ will be computed level-by-level from the grid layout.  We construct each level in grid patches of $32^3$ zones each.  
Consistent patch size allows us to optimize the memory usage and performance of the simulation.
The finest level is a thin pancake of
$100\times100\times1$ grids. This covers the midplane with a
resolution of 12.5 pc for 
the entire disk, to the dust scale height of $\pm 100$pc and a side length of
20kpc.
Resolution increases by a factor of 2 each level.  The next two levels increase
the aspect ratio of the refined region ($54\times54\times3$ and
$30\times30\times11$ grids, respectively), and the remaining levels are $16\times 16 \times 16$
grids.   This grid structure gives an outer size of $8.2\sci{5}$pc on a side
with 12.5pc resolution over the whole disk at the midplane.  

The number of updates is found from the total simulation time over the step
size, $N_U=\Delta T/d t$. $T$ by the
problem goals, and we estimate $dt$ from the solver behavior and prior
results.
As described above, the \emph{turbulence} simulations will run for
 $T=5 t_{dyn} = 5/(2\mach)$ (in dimensionless code units), to statistically
resolve the turbulence. The \emph{galaxy} simulations will run for $T=2$ Gyr,
which represents roughly 8 orbits of the galaxy, hopefully enough time to grow
the magnetic field.  The timestep size, $dt$,
is found by a standard Courant condition,
\begin{align}
	\Delta t = \eta \frac{\Delta x}{(v+c_f)_{max}} \label{eqn.dt},
\end{align}
where $(v+c_f)_{max}$ is the maximum signal velocity on each resolution level.  Here, $v$ is the velocity
and $c_f$ is the fast MHD speed.  It is not possible to predict the exact value
of this signal speed, so we calibrate to lower resolution simulations and use
Equation \ref{eqn.dt} to rescale.  For the \emph{turbulence} simulations, we
calibrate $\Delta t$ to the fiducial simulations in Stalpes et al (2023).  For
the \emph{galaxy} simulations, we calibrate to a low-resolution preliminary
galaxy.

Table \ref{table3} shows a summary of the request.  The first porton show each
level of one  \emph{galaxy} simulation, estimating $N_Z$ and $N_U=\Delta T/d t$ and the total
cost, $SU$, for each level $\ell$.  The second portion shows the cost for each
turbulence simulation, broken down by Mach numbers which dictate both $\Delta T$
and $dt$.
